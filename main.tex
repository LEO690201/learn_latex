\documentclass{article}
%起始
\usepackage[utf8]{inputenc}
%语言设置
\usepackage{amsmath}
\usepackage{amssymb}
\usepackage{amsfonts}

\usepackage{geometry}
%自定义页面布局
\geometry{a4paper, margin=1in}

\usepackage{ctex}
%中文宏包
\usepackage{hyperref}
%创建链接
\usepackage{graphicx}
%插入图片

\usepackage{hyperref} 
% 用于生成带超链接的目录
\hypersetup{hidelinks, colorlinks=true,allcolors=black}
% 消除红色超链接框
\usepackage{tocloft} 
% 用于自定义目录格式

% 自定义目录格式(可选)
\renewcommand{\cftsecnumwidth}{3em} % 调整章节编号与标题之间的间距
\renewcommand{\cftsubsecnumwidth}{3em} % 调整小节编号与标题之间的间距


\newenvironment{proof}{{\noindent\it 证明}\quad}{\hfill $\square$\par}
%证明与结束证明

\usepackage{pgfplots} % 用于绘图
\pgfplotsset{compat=1.18} % 设置pgfplots的兼容性版本



\usepackage[backend=biber, style=numeric]{biblatex} % 选择你喜欢的样式
\addbibresource{references.bib} % 你的参考文献数据库文件



\title{Latex学习笔记}

\author{胡广理}
\date{2025.2.25}


%%%%%%%%%%%%%%%%%%%%%%%%%%%%%%%%%%



\begin{document}
\maketitle


% ewpage
\tableofcontents
%插入目录

%%%%%%%%%%%%%%%%%%%%%%%%%%%%%%%%
%以上复制即可(Xelatex)

\section{hgl-0}


\section{hgl-1}   % 最大的节号(顺序自动)


\section*{hgl-1}   % 自定义的节号


\subsection{HGL}   % 第二大的节号

\subsubsection{段落}   %第三大的节号
1234455   %只换行算回车
123
%空一行才算换行

0000

\subsubsection{证明}

\begin{proof}  %证明

1+1=2

2+2=4
\end{proof}

\subsubsection{大括号分类}

\[   %大括号分类
I_n = \begin{cases} 
K & \text{if } n \mod 2 = -1 \\
L & \text{if } n \mod 2 = 1 
\end{cases}.
\]

\subsubsection{按条分类(PPT里可能更常用)}

\begin{enumerate}
    \item 123
    \item 456
    \item 789
    \item[000] 5677  %不使用序号,使用自定义的符号作为序号
\end{enumerate}

\subsubsection{其他功能(不用记住,在工具栏都有)}
\begin{enumerate}
    \item 插入图片   %在上面有一个按键
    \begin{figure}
        \centering
        \includegraphics[width=0.5\linewidth]{屏幕截图 161805.png}
        \caption{这是一个篮球}
        \label{fig:enter-label}
    \end{figure}

    \item 插入表格   %图片和表格位置不固定,所以要有类似于"如图1"之类的标识
    
    \begin{table}
        \centering
        \begin{tabular}{ccc}
            1 & 3 & 2\\
            2 & 3 & 3\\
        \end{tabular}
        \caption{这是一个表格}
        \label{tab:my_label}
    \end{table}

    \item 插入链接
    
    \href{https://latex.ustc.edu.cn/}{这是中科大latex的链接}

    \item 插入行间数学公式
\[  
\left\|x\right\|^{2}+\left\|y\right\|^{2}
\]



其中:数学公式可以通过网址进行编辑
\href{https://www.latexlive.com/home}{网站链接}

注意:特殊符号:\[\setminus \hspace{10pt} \%\]

分别对应除号,空格,百分号

\end{enumerate}

\cite{li1975}
\cite{sharkovsky1964}
\cite{12345}.
\cite{67890}.


\printbibliography

\end{document}


